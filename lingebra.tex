\documentclass[a5paper,12pt]{amsbook}

\usepackage[czech]{babel}
\usepackage[utf8]{inputenc}
\usepackage[T1]{fontenc}
\usepackage{color}
\usepackage{svg}

\theoremstyle{definition}
\newtheorem{definition}{Definice}[chapter]
\newtheorem{example}{Příklad}[chapter]

% Book styles
\newcommand{\myscalar}[1]{#1}
\newcommand{\myvec}[1]{\mathbf{#1}}
\newcommand{\mycoord}[1]{\overrightarrow{\mathbf{#1}}}
\newcommand{\mymatrix}[1]{\mathbf{#1}}
\newcommand{\myspace}[1]{\mathbb{#1}}
\newcommand{\mymap}[1]{#1}

\begin{document}

\title{Lineární algebra pro debily}
\author{Ondřej Stárek}
\maketitle

\chapter{Úvod}

\noindent Tento text se snaží čtenáře uvést do problematiky některé pokročilejší lineární algebry.
Je vedena pocitem autora, že pokud překročíte hranici prvního kurzu algebry (za vektorové
prostory), učitelé zapomněli učit a utápí se v nadšené záplavě písmenek a symbolů, namísto
vysvětlování souvislostí.

Text si neklade žádné nároky na matematickou přesnost. Koneckonců nenapsal ho matematik,
ale programátor, který se samostudiem pokoušel naučit něco nového. Spíše se jedná o poznámky
a popsané úvahy, které musel udělat, když se prokousával spoustou matematických textů
a pokoušel se je pochopit. Vlastně tímto textem danou látku vysvětluje především sám sobě.
Přesto si autor fandí, že pro jiné samouky by text mohl být přínosem a ulehčí jim cestu
k pochopení.

Text je určený samoukům, kteří už v současné době znají základy lineární algebry: maticové
operace, tělesa a vektorové prostory.

\chapter{Lineární zobrazení}

\section{Definice}

\noindent Lineární zobrazení je všeobecně známý pojem, obvykle probíraný již v základním kurzu
algebry. Nicméně i u něho autorovi zůstala řada nejasností. Proto se mu bude věnovat.

Co vlastně takové lineární zobrazení je? Není na tom nic složitého: jedná se
o zobrazení/transformaci/převod vektorů z jednoho vektorového prostoru do jiného. Zobrazení ovšem
není libovolné - musí zachovávat strukturu vektorového prostoru. To znamená, že pokud nějaký vektor
ve zdrojovém prostoru vynásobím skalárem, obrazem násobku bude stejný násobek jeho obrazu. Pokud
sečtu dva vektory, obrazem součtu bude součet jejich obrazů. 

Pokud myšlenka stále není jasná, zkusme to na obrázku. Zde je ukázka transformace, která prodlužuje
měřítko na vodorovné ose. Na levém obrázku je původní součet dvou vektorů. Vpravo jsou všechny vektory
přetransformované. Je vidět, že vůbec nezáleží, zda vektory nejdříve sečteme, a pak transformujeme,
anebo nejdříve transformujeme, a pak sečteme.

\begin{center}
\includesvg[width=200pt]{lintrans1}
\end{center}

\noindent Požadovat po lineárním zobrazení tyto podmínky je poměrně přirozené. Díky nim cokoliv, co můžeme
prohlásit o vektorech ve zdrojovém prostoru, platí zároveň i na jejich obrazech.

Nyní víme, co bychom od lineárního zobrazení očekávali, zkusme ho tedy přesněji nadefinovat.

\begin{definition}
Mějme vektorové prostory $\myspace{V}$ a $\myspace{W}$ oba nad tělesem $\myspace{T}$ a zobrazení
$\mymap{L}: \myspace{V}\rightarrow\myspace{W}$. Zobrazení $\mymap{L}$ je linerání zobrazení právě tehdy
pokud platí:

\begin{enumerate}
  \item $\forall\myvec{x}\in\myspace{V},\forall\myscalar{\alpha}\in\myspace{T}\;
    \mymap{L}(\myscalar{\alpha}\myvec{x})=\myscalar{\alpha}L(\myvec{x})$
  \item $\forall\myvec{x},\myvec{y}\in\myspace{V}\;\mymap{L}(\myvec{x}+\myvec{y})
    =\mymap{L}(\myvec{x})+\mymap{L}(\myvec{y})$
\end{enumerate}

\end{definition}

\noindent Definice je v principu jednoduchá, ale protože ďábel tkví v detailech, probereme ji podrobněji:

\begin{itemize}
  \item povšimněte si, že oba vektorové prostory jsou nad stejným tělesem. To je nutnost, jinak bychom
    nebyli schopní obraz vektoru násobit stejným skalárem. Linerní zobrazení tedy jsou omezená pouze
    na prostory nad stejným tělesem.
  \item V zápisu jsou použité stejné značky pro sčítání vektorů a pro násobení skalárem. Ale to neznamená,
    že se vždy jedná o stejné operace. V některých případech se jedná o operace prostoru $\myspace{V}$
    a v jiném o operace prostoru $\myspace{W}$. Nechávám na čtenáři, ať si odvodí, kdy se jedná o kterou
    (nehledejte v tom žádný chyták).
  \item Obě podmínky jsou poměrně přímočaré. První říká, že lineární zobrazení zachovává operaci násobení
    skalárem, druhá, že zachovává operaci sčítání vektorů.
\end{itemize}

Abych neskončil jenom u písmenek, ukažme si několik příkladů.

\begin{example}\textbf{Identita} - jedno z nejjednodušších zobrazení je takové, které převádí vektor
na ten samý:
\begin{equation*}
\mymap{L}:\,\myspace{V}\rightarrow\myspace{V}:\;\mymap{L}(\myvec{x})=\myvec{x}
\end{equation*} 
Je triviální si dokázat, že obě podmínky definice jsou zachované. Aby ne, když původní vektor
a jeho obraz jsou stejné.

Identita je zároveň ukázkou typické situace. Ačkoliv definice lineárního zobrazení pracuje se dvěma
prostory, typické použití je zobrazení do stejného prostoru, tedy kdy oba prostory jsou stejné.

\end{example}

\begin{example}\label{example:rotate}\textbf{Rotace} vektoru o nějaký úhel je dalším běžným příkladem
lineárního zobrazení:
\begin{center}
\includesvg[width=200pt]{lintrans2}
\end{center}
Rotace nemění délku vektorů, pouze ho pootočí. První podmínka tedy splněna je - nezávisí zda
vektor prodloužíme před nebo až po transformaci. Druhá podmínka je graficky také splněna - oba zdrojové
vektory i jejich součet se otočí stejně. Formální důkaz si čtenář určitě dokáže udělat sám.

\end{example}

\begin{example}\label{example:polynoms}\textbf{Substituce v polynomu}. Zkusme nyní ne úplně tradiční
vektorový prostor polynomů do řádu 3. Vektory prostoru tedy jsou funkce typu
\begin{equation*}
p(x)=ax^{3}+bx^{2}+cx+d
\end{equation*}
Operace násobení skalárem a sčítání polynomů jsou definované tak, jak jsme u polynomů zvyklí.
Nyní proveďme substituci $x=\frac{t}{2}$
\begin{equation*}
p(\frac{t}{2})=a\left(\frac{t}{2}\right)^{3}+b\left(\frac{t}{2}\right)^{2}+c\left(\frac{t}{2}\right)+d 
  = \frac{a}{8}t^{3}+\frac{b}{4}t^{2}+\frac{c}{2}t+d
\end{equation*} 
\noindent Je vidět, že výsledek substituce je opět polynom řádu 3. Tedy tato substituce funguje
jako zobrazení z našeho prostoru do našeho prostoru. Jedná se o lineární zobrazení?
\begin{enumerate}
  \item Obrazem vektoru $\alpha p(x)$ je
    \begin{equation*}
      \alpha\frac{a}{8}t^{3}+\alpha\frac{b}{4}t^{2}+\alpha\frac{c}{2}t+\alpha{}d 
      = \alpha p(\frac{t}{2})
    \end{equation*}
  \item Obrazem vektoru $p_1(x) + p_2(x)$ je
    \begin{equation*}
      \frac{a_1+a_2}{8}t^{3}+\frac{b_1+b_2}{4}t^{2}+\frac{c_1+c_2}{2}t+d_1+d_2
      = p_1(\frac{t}{2})+p_2(\frac{t}{2})
    \end{equation*}
\end{enumerate}
Obě podmínky definice lineárního zobrazení jsou zřejmě splněné, takže substituce
je lineárním zobrazením.

K čemu nám něco takového je? Například tato substituce je základem velmi elegantního algoritmu
pro kreslení křivek, který si parametrickou rovnici $p(t)=0$ pomocí této substituce rozkrájí
na drobné úseky tak, aby na každém z nich byl rozdíl mezi $p(0)$ a $p(1)$ právě jeden pixel
zobrazovacího zařízení.

\end{example}

\section{Souřadnice a báze}

\noindent Ze základního kurzu lineární algebry víme, že každý vektor nějakého vektorového prostoru lze
vyjádřit jako lineární kombinaci bázových vektorů (koeficienty lineární kombinace se nazývají
\textit{souřadnice}). Ačkoliv báze mohou být různé, počet bázových vektorů je vždy stejný
a definuje \textit{dimenzi vektorového prostoru}. Omezme se nyní na prostory s konečnou dimenzí
a zkusme zapřemýšlet o vyjádření lineárního zobrazení v souřadnicích.

Vezměme si tedy zdrojový vektorový prostor $\myspace{V}$ dimenze $n$ a jeho bázi 
$\{\myvec{b_{v_1}},\myvec{b_{v_2}},\ldots,\myvec{b_{v_n}}\}$. Stejně tak cílový
vektorový prostor $\myspace{W}$ dimenze $m$ a jeho bázi $\{\myvec{b_{w_1}}, \myvec{b_{w_2}},
\ldots,\myvec{b_{w_m}}\}$. A mějme lineární zobrazení $\mymap{L}: \myspace{V}\rightarrow\myspace{W}$
mezi nimi. Víme, že každý vektor $\myvec{v}\in\myspace{V}$ lze vyjádřit jako lineární kombinaci
\begin{equation*}
\myvec{v}=\myscalar{v_{1}}\myvec{b_{v_1}}+\myscalar{v_{2}}\myvec{b_{v_2}}+\cdots+\myscalar{v_{n}}
  \myvec{b_{v_n}}
\end{equation*} 
Pokud tento vektor použijeme jako argument lineárního zobrazení, tak za použití obou podmínek
z definice:
\begin{equation*}
\begin{split}
\myvec{w}=L(\myvec{v})&=\mymap{L}(\myscalar{v_{1}}\myvec{b_{v_1}}+\myscalar{v_{2}}\myvec{b_{v_2}}
    +\cdots+\myscalar{v_{n}}\myvec{b_{v_n}})\\
  &=\myscalar{v_{1}}\mymap{L}(\myvec{b_{v_1}})+\myscalar{v_{2}}\mymap{L}(\myvec{b_{v_2}})+\cdots
    +\myscalar{v_{n}}\mymap{L}(\myvec{b_{v_n}})
\end{split}
\end{equation*} 
Tedy obraz vektoru $\myvec{v}$ je lineární kombinace obrazů bázových vektorů.
Pokud i obrazy bázových vektorů vyjádříme v souřadnicích prostoru $\myspace{W}$, lze lineární
zobrazení zapsat pomocí maticové aritmetiky (tzn. obrazy vektorů báze v předchozí rovnici
vyjádříme jako souřadnice v prostoru $\myspace{W}$ a sčítáním po složkách získáme souřadnice
vektoru $\myvec{w}$):
\begin{equation*}
\left(\begin{array}{c}
w_{1}\\
w_{2}\\
\vdots\\
w_{m}
\end{array}\right)=\left(\begin{array}{cccc}
\mymap{L}(\myvec{b_{v_1}})_1 & \mymap{L}(\myvec{b_{v_2}})_1 & \cdots & \mymap{L}(\myvec{b_{v_n}})_1\\
\mymap{L}(\myvec{b_{v_1}})_2 & \mymap{L}(\myvec{b_{v_2}})_2 & \cdots & \mymap{L}(\myvec{b_{v_n}})_2\\
\vdots & \vdots & \ddots & \vdots\\
\mymap{L}(\myvec{b_{v_1}})_m & \mymap{L}(\myvec{b_{v_2}})_m & \cdots & \mymap{L}(\myvec{b_{v_n}})_m
\end{array}\right)\left(\begin{array}{c}
v_{1}\\
v_{2}\\
\vdots\\
v_{n}
\end{array}\right)
\end{equation*}
nebo zjednodušeně:
\begin{equation*}
\mycoord{w}=\mymatrix{L}\mycoord{v}
\end{equation*}
kde sloupce matice $\mymatrix{L}$ jsou souřadnice obrazů zdrojové báze a $\mycoord{v}$
a $\mycoord{w}$ jsou souřadnice vektorů $\myvec{v}$ a $\myvec{w}$.

\begin{example}Pokračujme v příkladu \ref{example:rotate}. Pokud vektor pootočíme o úhel $\alpha$,
situace bude vypadat takto:
\begin{center}
\includesvg[width=200pt]{rotate}
\end{center}
Souřadnice původního vektoru jsou
\begin{align*} 
x &= d\cos\beta\\ 
y &= d\sin\beta
\end{align*}
kde $d$ je délka vektoru. Souřadnice transformovaného vektoru jsou
\begin{align*} 
x' &= d\cos(\beta+\alpha) = d\cos\beta\cos\alpha - d\sin\beta\sin\alpha\\ 
y' &= d\sin(\beta+\alpha) = d\sin\beta\cos\alpha + d\cos\beta\sin\alpha
\end{align*}
Když dosadíme za $d$ z původních souřadnic dostáváme
\begin{align*} 
x' &= x\cos\alpha - y\sin\alpha\\ 
y' &= x\sin\alpha + y\cos\alpha
\end{align*}
V maticovém tvaru rotaci vyjádříme takto:
\begin{equation*}
\left(\begin{array}{cc}
\cos\alpha & -\sin\alpha\\
\sin\alpha & \cos\alpha\\
\end{array}\right)\left(\begin{array}{c}
x\\
y
\end{array}\right)=\left(\begin{array}{c}
x'\\
y'
\end{array}\right)
\end{equation*}
Transformace v této podobě je používaná například v počítačové grafice. Grafik
vytvoří model, aniž by znal jeho umístění ve scéně. Ten, kdo scénu kompletuje,
model někam umístí a otočí si ho podle potřeby.

Otočení souřadnicového systému je samozřejmě také běžná činnost ve fyzice.

\end{example}

\begin{example}Vraťme se k příkladu \ref{example:polynoms} a vezměme si jednoduchou bázi prostoru
$\{x^3, x^2, x, 1\}$, polynom $p(x)$ vyjádřený jako souřadnice je $(a, b, c, d)^T$ a matice zobrazení
vypadá takto:
\begin{equation*}
\left(\begin{array}{cccc}
\frac{1}{8} & 0 & 0 & 0\\
0 & \frac{1}{4} & 0 & 0\\
0 & 0 & \frac{1}{2} & 0\\
0 & 0 & 0 & 1
\end{array}\right)\left(\begin{array}{c}
a\\
b\\
c\\
d
\end{array}\right)=\left(\begin{array}{c}
\frac{1}{8}a\\
\frac{1}{4}b\\
\frac{1}{2}c\\
d
\end{array}\right)
\end{equation*}
Ptáte se, k čemu nám to je? Už jsem zmínil, že tato substituce je základ pro jeden
algoritmus. Jenže ouha, počítače neumí jednoduše kouknout na nějaký vzoreček a do
něho substituovat. Co ale umí skvěle, je násobit matice. Tím, že jsme polynom vyjádřili
jako souřadnice a substituci jako matici, jsme získali snadný způsob, jak algoritmus
naimplementovat.
\end{example}

Nyní víme, jak lineární zobrazení vyjádřit jako matici. Protože každý vektor lze vyjádřit
jako lineární kombinaci bázových vektorů, jsme matici schopni vytvořit vždy (skládá se ze
souřadnic obrazů zdrojové báze). Tzn. \textbf{pro každé lineární zobrazení} (stále uvažujeme
prostory s konečnou dimenzí) \textbf{existuje jeho matice}.

V knížkách algebry se už obvykle příliš nezdůrazňuje, že předchozí platí i \textbf{obráceně}, tedy 
že \textbf{každá dvourozměrná matice určuje nějaké lineární zobrazení.}

Každý vektor lze jednoznačně vyjádřit jako souřadnice. I opačně každé souřadnice odpovídají nějakému
vektoru (obsahují koeficienty lineární kombinace báze a z definice vektorového prostoru je každá
lineární kombinace vektorů opět vektor). Tudíž, pokud matici $\mymatrix{A}$ vynásobíme souřadnicemi
libovolného vektoru, získáme opět souřadnice nějakého vektoru. A protože z definice maticových
operací platí
\begin{enumerate}
  \item$
      \mymatrix{A}\left(\myscalar{\alpha}\mycoord{x}\right)=
          \myscalar{\alpha}\left(\mymatrix{A}\mycoord{x}\right)$,
  \item$
      \mymatrix{A}\left(\mycoord{x_1}+\mycoord{x_2}\right)=
          \mymatrix{A}\mycoord{x_1}+\mymatrix{A}\mycoord{x_2}$,
\end{enumerate}
jsou splněné obě podmínky definice lineárního zobrazení.

Povšiměte si, že v předchozím vůbec neříkám, o jaké konkrétní prostory se jedná. Matice tedy
určuje obecně nekonečně mnoho lineárních zobrazení mezi libovolnými vektorovými prostory dimenzí
odpovídajících rozměrům matice (samozřejmě matice i prostory musí být nad stejným tělesem).

I když se jedná o poměrně triviální fakt, jde o podstatnou myšlenkovou úvahu.
Autor měl například dlouho problém pochopit termín \textit{podobnost matic}. Matice je jenom
pole čísel, proto podobnost nedává žádný smysl, obzvláště pokud podobné matice mají zcela
rozdílná čísla. Jejich podobnost spočívá v tom, že odpovídají stejnému lineárnímu zobrazení
v různých bázích (transformace bází viz. kapitola XXX). I další pojmy algebry, například vlastní
čísla matice, dávají smysl až s tímto pozorováním. Matematici často mezi oběma pojmy tiše přechází,
což před pochopením této souvislosti samoukům, jako jsem já, způsobuje bolení hlavy.

\section{Transformace bází}

\noindent Zkusme nyní zapřemýšlet, jak je možné transformovat souřadnicové systémy.  K čemu by nám
něco takového bylo? Transformace souřadnicových systémů je důležitá pro fyziku. Platnost
fyzikálních zákonů musí být nezávislá na souřadnicích. Jednoduše řečeno Newtonovo jablko bude
pořád padat zcela stejně, bez ohledu na to, jak si zvolíme osy. Pro výpočty i měření si ovšem
nějaké zvolit musíme a musíme také být schopní číselné výsledky mezi souřadnicovými systémy
převádět.

Souřadnice vektorů jsou určené bází, tedy otázku můžeme přeformulovat na transformaci bází.
Potřebujeme tedy zjistit, jaké podmínky musí lineární zobrazení splňovat, aby transformací
báze zdrojového vektorového prostoru vznikla opět báze cílového prostoru. Je zřejmé, že ne každé
lineární zobrazení toto splňuje. Například zobrazení, které každý vektor transformuje na nulový
vektor, nedokáže vytvořit bázi, protože množina nulových vektorů není lineárně nezávislá.

Mějme tedy opět lineární zobrazení $\mymap{L}: \myspace{V}\rightarrow\myspace{W}$, kde $\myspace{V}$
a $\myspace{W}$ jsou vektorové prostory konečných dimenzí $n > 0$ a $m > 0$, a bázi prostoru $\myspace{V}$
$\{\myvec{v_1}, \myvec{v_2}, \ldots, \myvec{v_n}\}$. Za jakých podmínek bude množina obrazů báze
$\{\mymap{L}(\myvec{v_1}), \mymap{L}(\myvec{v_2}), \ldots, \mymap{L}(\myvec{v_n})\}$ bází prostoru
$\myspace{W}$? Z definice báze
\begin{enumerate}
  \item musí generovat celý prostor $\myspace{W}$,
  \item musí být lineárně nezávislá.
\end{enumerate}
První podmínku lze splnit poměrně snadno. Znamená, že každý prvek prostoru $\myspace{W}$ musí být
lineární kombinací obrazů báze, a z definice lineárního zobrazení musí být obrazem nějakého
vektoru prostoru $\myspace{V}$:
\begin{equation*}
\begin{split}
\myvec{w}
  &= \myscalar{\alpha{}_1}\mymap{L}(\myvec{v_1}) + \myscalar{\alpha{}_2}\mymap{L}(\myvec{v_2})
      + \cdots + \myscalar{\alpha{}_n}\mymap{L}(\myvec{v_n}) \\
  &= \mymap{L}(\myscalar{\alpha{}_1}\myvec{v_1} + \myscalar{\alpha{}_2}\myvec{v_2} + \cdots 
      + \myscalar{\alpha{}_n}\myvec{v_n}) = \mymap{L}(\myvec{v})
\end{split}
\end{equation*}
Lineární zobrazení tedy musí být \textbf{na} (stručné označení, že každý prvek prostoru 
$\myspace{W}$ je obrazem nějakého prvku prostoru $\myspace{V}$), protože pak je možné
každý prvek prostoru $\myspace{W}$ vyjádřit jako lineární kombinaci obrazů báze. Pokud
by nebylo \textbf{na}, znamená to, že existuje vektor, který nelze vyjádřit jako lineární
kombinaci obrazů báze, tudíž množina obrazů báze negeneruje celý prostor $\myspace{W}$.

Druhá podmínka je složitější. Pokud je množina obrazů báze lineárně závislá, pak existuje
nenulová lineární kombinace
\begin{equation*}
\myscalar{\alpha{}_1}\mymap{L}(\myvec{v_1}) + \myscalar{\alpha{}_2}\mymap{L}(\myvec{v_2}) 
  + \cdots + \myscalar{\alpha{}_n}\mymap{L}(\myvec{v_n}) = \myvec{0}
\end{equation*}
 Z podmínky (1) víme, že množina generuje celý prostor $\myspace{W}$, proto je
možné každý vektor $\myvec{w}\neq\myvec{0}\in\myspace{W}$ vyjádřit jako nenulovou lineární kombinaci
\begin{equation*}
\begin{split}
\myvec{w} &= \myscalar{\beta{}_1}\mymap{L}(\myvec{v_1}) + \myscalar{\beta{}_2}\mymap{L}(\myvec{v_2}) 
  + \cdots + \myscalar{\beta{}_n}\mymap{L}(\myvec{v_n}) \\
&= \mymap{L}(\myscalar{\beta{}_1}\myvec{v_1} + \myscalar{\beta{}_2}\myvec{v_2} 
  + \cdots + \myscalar{\beta{}_n}\myvec{v_n})
\end{split}
\end{equation*}
K tomuto vyjádření vektoru $\myvec{w}$ je možné libovolně přičítat první kombinaci obrazů báze
(přičítání nulového vektoru nic nezmění):
\begin{equation*}
\begin{split}
\myvec{w} + \myvec{0} &= \myscalar{\beta{}_1}\mymap{L}(\myvec{v_1}) 
  + \myscalar{\beta{}_2}\mymap{L}(\myvec{v_2}) + \cdots + \myscalar{\beta{}_n}\mymap{L}(\myvec{v_n}) \\
  &+ \myscalar{\alpha{}_1}\mymap{L}(\myvec{v_1}) + \myscalar{\alpha{}_2}\mymap{L}(\myvec{v_2}) 
  + \cdots + \myscalar{\alpha{}_n}\mymap{L}(\myvec{v_n}) \\
&= \mymap{L}((\myscalar{\beta{}_1} + \myscalar{\alpha{}_1})\myvec{v_1} 
  + (\myscalar{\beta{}_2} + \myscalar{\alpha{}_2})\myvec{v_2} 
  + \cdots + (\myscalar{\beta{}_n} + \myscalar{\alpha{}_n})\myvec{v_n})
\end{split}
\end{equation*}
Protože obě kombinace jsou nenulové, je zřejmé, že vektor $\myvec{w}$ je obrazem dvou různých
vektorů z prostoru $\myspace{V}$ (jejich souřadnice jsou rozdílné). Znamená to tedy, že
linerání zobrazení \textbf{není prosté}. Tedy, pokud zobrazení \textbf{je prosté}, pak je množina
obrazů báze lineárně nezávislá\footnote{
  Pokud jste se v poslední větě ztratili, jedná se o tzv. \textit{nepřímý důkaz}, kdy se
  implikace $A\Rightarrow B$ dokazuje jako $\lnot B\Rightarrow\lnot A$. Já jsem dokázal, že pokud
  množina obrazů je lineárně závislá, pak zobrazení není prosté. Tedy obráceně pokud zobrazení
  je prosté, množina je lineárně nezávislá.
}.

Získali jsme tedy dvě nutné a zároveň postačující podmínky. Lineární zobrazení, které 
je \textbf{prosté} a \textbf{na} transformuje bázi na jinou bázi a tedy slouží jako transformace
souřadnicového systému.

Otázka zní, zda je možné tvrdit i opak: každé lineární zobrazení prosté a na transformuje bázi.
I toto platí, ačkoliv důkaz zde dělat nebudu. Lze udělat podobným způsobem a čtenář ho najde
určitě podstatně elegantnější podobě v učebnicích algebry.

Položmě si nyní otázku: mohou být dimenze prostoru $\myspace{V}$ a $\myspace{W}$ rozdílné?
Odpověď zní nikoliv, nemohou:
\begin{enumerate}
  \item $m < n$: obrazů báze je více, než je dimenze prostoru $\myspace{W}$, tudíž obrazy
      nemohou být lineárně nezávislé. A proto ani zobrazení nemůže být \textbf{prosté}.
  \item $m > n$: obrazů báze je méně, tudíž musí existovat další vektor, který nelze vyjádřit
      jako lineární kombinaci obrazů báze. Tedy zobrazení není \textbf{na}.
\end{enumerate}

Nyní tedy víme, že lineární zobrazení prosté a na funguje jako transformace báze. Takové
zobrazení se nazývá \textit{bijekce} a má jednu fajn vlastnost: existuje pro něj inverzní
zobrazení. To tedy znamená, že pokud přetransformuji souřadnicový systém, vždy jsem schopen
se vrátit zpět k původním souřadnicím.

\end{document}


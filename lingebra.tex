\documentclass[a5paper,12pt]{amsbook}

\usepackage[czech]{babel}
\usepackage[utf8]{inputenc}
\usepackage[T1]{fontenc}
\usepackage{color}
\usepackage{svg}

\theoremstyle{definition}
\newtheorem{definition}{Definice}[chapter]
\newtheorem{example}{Příklad}[chapter]

\begin{document}

\title{Lineární algebra pro debily}
\author{Ondřej Stárek}
\maketitle

\chapter{Úvod}

\noindent Tento text se snaží čtenáře uvést do problematiky některé pokročilejší lineární algebry.
Je vedena pocitem autora, že pokud překročíte hranici prvního kurzu algebry (za vektorové
prostory), učitelé zapomněli učit a utápí se v nadšené záplavě písmenek a symbolů, namísto
vysvětlování souvislostí.

Text si neklade žádné nároky na matematickou přesnost. Koneckonců nenapsal ho matematik,
ale programátor, který se samostudiem pokoušel naučit něco nového. Spíše se jedná o poznámky
a popsané úvahy, které musel udělat, když se prokousával spoustou matematických textů
a pokoušel se je pochopit. Vlastně tímto textem danou látku vysvětluje především sám sobě.
Přesto si autor fandí, že pro jiné samouky by text mohl být přínosem a ulehčí jim cestu
k pochopení.

Text je určený samoukům, kteří už v současné době znají základy lineární algebry: maticové
operace, tělesa a vektorové prostory.

\chapter{Lineární zobrazení}

\section{Definice}

\noindent Lineární zobrazení je všeobecně známý pojem, obvykle probíraný již v základním kurzu
algebry. Nicméně i u něho autorovi zůstala řada nejasností. Proto se mu bude věnovat.

Co vlastně takové lineární zobrazení je? Není na tom nic složitého: jedná se
o zobrazení/transformaci/převod vektorů z jednoho vektorového prostoru do jiného. Zobrazení ovšem
není libovolné - musí zachovávat strukturu vektorového prostoru. To znamená, že pokud nějaký vektor
ve zdrojovém prostoru vynásobím skalárem, obrazem násobku bude stejný násobek jeho obrazu. Pokud
sečtu dva vektory, obrazem součtu bude součet jejich obrazů. 

Pokud myšlenka stále není jasná, zkusme to na obrázku. Zde je ukázka transformace, která prodlužuje
měřítko na vodorovné ose. Na levém obrázku je původní součet dvou vektorů. Vpravo jsou všechny vektory
přetransformované. Je vidět, že vůbec nezáleží, zda vektory nejdříve sečteme, a pak transformujeme,
anebo nejdříve transformujeme, a pak sečteme.

\begin{center}
\includesvg[width=200pt]{lintrans1}
\end{center}

\noindent Požadovat po lineárním zobrazení tyto podmínky je poměrně přirozené. Díky nim cokoliv, co můžeme
prohlásit o vektorech ve zdrojovém prostoru, platí zároveň i na jejich obrazech.

Nyní víme, co bychom od lineárního zobrazení očekávali, zkusme ho tedy přesněji nadefinovat.

\begin{definition}
Mějme vektorové prostory $\mathbb{V}$ a $\mathbb{W}$ oba nad tělesem $\mathbb{T}$ a zobrazení $L: \mathbb{V}\rightarrow\mathbb{W}$. Zobrazení $L$ je linerání zobrazení právě tehdy pokud platí:

\begin{enumerate}
  \item $\forall\mathbf{x}\in\mathbb{V},\forall\alpha\in\mathbb{T}\;L(\alpha\mathbf{x})=\alpha L(\mathbf{x})$
  \item $\forall\mathbf{x},\mathbf{y}\in\mathbb{V}\;L(\mathbf{x}+\mathbf{y})=L(\mathbf{x})+L(\mathbf{y})$
\end{enumerate}

\end{definition}

\noindent Definice je v principu jednoduchá, ale protože ďábel tkví v detailech, probereme ji podrobněji:

\begin{itemize}
  \item povšimněte si, že oba vektorové prostory jsou nad stejným tělesem. To je nutnost, jinak bychom
    nebyli schopní obraz vektoru násobit stejným skalárem. Linerní zobrazení tedy jsou omezená pouze
    na prostory nad stejným tělesem.
  \item V zápisu jsou použité stejné značky pro sčítání vektorů a pro násobení skalárem. Ale to neznamená,
    že se vždy jedná o stejné operace. V některých případech se jedná o operace prostoru $\mathbb{V}$
    a v jiném o operace prostoru $\mathbb{W}$. Nechávám na čtenáři, ať si odvodí, kdy se jedná o kterou
    (nehledejte v tom žádný chyták).
  \item Obě podmínky jsou poměrně přímočaré. První říká, že lineární zobrazení zachovává operaci násobení
    skalárem, druhá, že zachovává operaci sčítání vektorů.
\end{itemize}

Abych neskončil jenom u písmenek, ukažme si několik příkladů.

\begin{example}\textbf{Identita} - jedno z nejjednodušších zobrazení je takové, které převádí vektor
na ten samý:
\begin{equation*}
L:\,\mathbb{V}\rightarrow\mathbb{V}:\;L(\mathbf{x})=\mathbf{x}
\end{equation*} 
Je triviální si dokázat, že obě podmínky definice jsou zachované. Aby ne, když původní vektor
a jeho obraz jsou stejné.

Identita je zároveň ukázkou typické situace. Ačkoliv definice lineárního zobrazení pracuje se dvěma
prostory, typické použití je zobrazení do stejného prostoru, tedy kdy oba prostory jsou stejné.

\end{example}

\begin{example}\label{example:rotate}\textbf{Rotace} vektoru o nějaký úhel je dalším běžným příkladem
lineárního zobrazení:
\begin{center}
\includesvg[width=200pt]{lintrans2}
\end{center}
Rotace nemění délku vektorů, pouze ho pootočí. První podmínka tedy splněna je - nezávisí zda
vektor prodloužíme před nebo až po transformaci. Druhá podmínka je graficky také splněna - oba zdrojové
vektory i jejich součet se otočí stejně. Formální důkaz si čtenář určitě dokáže udělat sám.

\end{example}

\begin{example}\label{example:polynoms}\textbf{Substituce v polynomu}. Zkusme nyní ne úplně tradiční
vektorový prostor polynomů do řádu 3. Vektory prostoru tedy jsou funkce typu
\begin{equation*}
p(x)=ax^{3}+bx^{2}+cx+d
\end{equation*}
Operace násobení skalárem a sčítání polynomů jsou definované tak, jak jsme u polynomů zvyklí.
Nyní proveďme substituci $x=\frac{t}{2}$
\begin{equation*}
p(\frac{t}{2})=a\left(\frac{t}{2}\right)^{3}+b\left(\frac{t}{2}\right)^{2}+c\left(\frac{t}{2}\right)+d 
  = \frac{a}{8}t^{3}+\frac{b}{4}t^{2}+\frac{c}{2}t+d
\end{equation*} 
\noindent Je vidět, že výsledek substituce je opět polynom řádu 3. Tedy tato substituce funguje
jako zobrazení z našeho prostoru do našeho prostoru. Jedná se o lineární zobrazení?
\begin{enumerate}
  \item Obrazem vektoru $\alpha p(x)$ je
    \begin{equation*}
      \alpha\frac{a}{8}t^{3}+\alpha\frac{b}{4}t^{2}+\alpha\frac{c}{2}t+\alpha{}d 
      = \alpha p(\frac{t}{2})
    \end{equation*}
  \item Obrazem vektoru $p_1(x) + p_2(x)$ je
    \begin{equation*}
      \frac{a_1+a_2}{8}t^{3}+\frac{b_1+b_2}{4}t^{2}+\frac{c_1+c_2}{2}t+d_1+d_2
      = p_1(\frac{t}{2})+p_2(\frac{t}{2})
    \end{equation*}
\end{enumerate}
Obě podmínky definice lineárního zobrazení jsou zřejmě splněné, takže substituce
je lineárním zobrazením.

K čemu nám něco takového je? Například tato substituce je základem velmi elegantního algoritmu
pro kreslení křivek, který si parametrickou rovnici $p(t)=0$ rozkrájí na drobné úseky
tak, aby na každém z nich byl rozdíl mezi $t=0$ a $t=1$ právě jeden pixel zobrazovacího zařízení.

\end{example}

\section{Souřadnice a báze}

\noindent Ze základního kurzu lineární algebry víme, že každý vektor nějakého vektorového prostoru lze
vyjádřit jako lineární kombinaci bázových vektorů (koeficienty lineární kombinace se nazývají
\textit{souřadnice}). Ačkoliv báze mohou být různé, počet bázových vektorů je vždy stejný
a definuje \textit{dimenzi vektorového prostoru}. Omezme se nyní na prostory s konečnou dimenzí
a zkusme zapřemýšlet o vyjádření lineárního zobrazení v souřadnicích.

Vezměme si tedy zdrojový vektorový prostor $\mathbb{V}$ dimenze $n$ a jeho bázi 
$\mathbf{b}_{V}=\{b_{v_1},b_{v_2},\ldots,b_{v_n}\}$. Stejně tak cílový vektorový prostor $\mathbb{W}$
dimenze $m$ a jeho bázi $\mathbf{b}_{W}=\{b_{w_1},b_{w_2},\ldots,b_{w_m}\}$. A mějme lineární zobrazení
$L: \mathbb{V}\rightarrow\mathbb{W}$ mezi nimi. Víme, že každý vektor $\mathbf{v}\in\mathbb{V}$
lze vyjádřit jako lineární kombinaci
\begin{equation*}
\mathbf{v}=v_{1}b_{v_1}+v_{2}b_{v_2}+\cdots+v_{n}b_{v_n}
\end{equation*} 
Pokud tento vektor použijeme jako argument lineárního zobrazení, tak za použití obou podmínek
z definice:
\begin{equation*}
\begin{split}
L(\mathbf{v})&=L(v_{1}b_{v_1}+v_{2}b_{v_2}+\cdots+v_{n}b_{v_n})\\
  &=v_{1}L(b_{v_1})+v_{2}L(b_{v_2})+\cdots+v_{n}L(b_{v_n})
\end{split}
\end{equation*} 
Tedy obraz vektoru $\mathbf{v}$ je lineární kombinace obrazů bázových vektorů. Pokud i obrazy bázových
vektorů vyjádříme v souřadnicích prostoru $\mathbb{W}$, lze lineární zobrazení zapsat pomocí maticové
aritmetiky (násobení skalárem a sčítání vektorů lze dělat po složkách souřadnic):
\begin{equation*}
\left(\begin{array}{c}
w_{1}\\
w_{2}\\
\vdots\\
w_{m}
\end{array}\right)=\left(\begin{array}{cccc}
a_{11} & a_{21} & \cdots & a_{n1}\\
a_{12} & a_{22} & \cdots & a_{n2}\\
\vdots & \vdots & \ddots & \vdots\\
a_{1m} & a_{2m} & \cdots & a_{nm}
\end{array}\right)\left(\begin{array}{c}
v_{1}\\
v_{2}\\
\vdots\\
v_{n}
\end{array}\right)
\end{equation*}
nebo zjednodušeně:
\begin{equation*}
\overrightarrow{\mathbf{w}}=\mathbf{L}\overrightarrow{\mathbf{v}}
\end{equation*}
kde sloupce matice $\mathbf{L}$ jsou souřadnice obrazů zdrojové báze a $\overrightarrow{\mathbf{v}}$
a $\overrightarrow{\mathbf{w}}$ jsou souřadnice vektorů $\mathbf{v}$ a $\mathbf{w}$.

\begin{example}Pokračujme v příkladu \ref{example:rotate}. Pokud vektor pootočíme o úhel $\alpha$,
situace bude vypadat takto:
\begin{center}
\includesvg[width=200pt]{rotate}
\end{center}
Souřadnice původního vektoru jsou
\begin{align*} 
x &= d\cos\beta\\ 
y &= d\sin\beta
\end{align*}
kde $d$ je délka vektoru. Souřadnice transformovaného vektoru jsou
\begin{align*} 
x' &= d\cos(\beta+\alpha) = d\cos\beta\cos\alpha - d\sin\beta\sin\alpha\\ 
y' &= d\sin(\beta+\alpha) = d\sin\beta\cos\alpha + d\cos\beta\sin\alpha
\end{align*}
Když dosadíme za $d$ z původních souřadnice dostáváme
\begin{align*} 
x' &= x\cos\alpha - y\sin\alpha\\ 
y' &= x\sin\alpha + y\cos\alpha
\end{align*}
V maticovém tvaru rotaci vyjádříme takto:
\begin{equation*}
\left(\begin{array}{cc}
\cos\alpha & -\sin\alpha\\
\sin\alpha & \cos\alpha\\
\end{array}\right)\left(\begin{array}{c}
x\\
y
\end{array}\right)=\left(\begin{array}{c}
x'\\
y'
\end{array}\right)
\end{equation*}
Transformace v této podobě je používaná například v počítačové grafice. Grafik
vytvoří model, aniž by znal jeho umístění ve scéně. Ten, kdo scénu kompletuje,
model někam umístí a otočí si ho podle potřeby.

Otočení souřadnicového systému je samozřejmě také běžná činnost ve fyzice.

\end{example}

\begin{example}Vraťme se k příkladu \ref{example:polynoms} a vezměme si jednoduchou bázi prostoru
$\{x^3, x^2, x, 1\}$, polynom $p(x)$ vyjádřený jako souřadnice je $(a, b, c, d)^T$ a matice zobrazení
vypadá takto:
\begin{equation*}
\left(\begin{array}{cccc}
\frac{1}{8} & 0 & 0 & 0\\
0 & \frac{1}{4} & 0 & 0\\
0 & 0 & \frac{1}{2} & 0\\
0 & 0 & 0 & 1
\end{array}\right)\left(\begin{array}{c}
a\\
b\\
c\\
d
\end{array}\right)=\left(\begin{array}{c}
\frac{1}{8}a\\
\frac{1}{4}b\\
\frac{1}{2}c\\
d
\end{array}\right)
\end{equation*}
Ptáte se, k čemu nám to je? Už jsem zmínil, že tato substituce je základ pro jeden
algoritmus. Jenže ouha, počítače neumí jednoduše kouknout na nějaký vzoreček a do
něho substituovat. Co ale umí skvěle, je násobit matice. Tím, že jsme polynom vyjádřili
jako souřadnice a substituci jako matici, jsme získali snadný způsob, jak algoritmus
naimplementovat.
\end{example}

Nyní víme, jak lineární zobrazení vyjádřit jako matici. Protože každý vektor lze vyjádřit
jako lineární kombinaci bázových vektorů, jsme matici schopni vytvořit vždy (skládá se ze
souřadnic obrazů zdrojové báze). Tzn. \textbf{pro každé lineární zobrazení} (stále uvažujeme
prostory s konečnou dimenzí) \textbf{existuje jeho matice}.

Co už se ovšem v knížkách algebry obvykle nezdůrazňuje, že předchozí platí i \textbf{obráceně}.
\textbf{Každá dvourozměrná matice definuje lineární zobrazení.} Z definice maticových operací
totiž platí:
\begin{enumerate}
  \item
    \begin{equation*}
      \mathbf{A}\left(\alpha\overrightarrow{\mathbf{x}}\right)=
          \alpha\left(\mathbf{A}\overrightarrow{\mathbf{x}}\right)
    \end{equation*}
  \item
    \begin{equation*}
      \mathbf{A}\left(\overrightarrow{\mathbf{x_1}}+\overrightarrow{\mathbf{x_2}}\right)=
          \mathbf{A}\overrightarrow{\mathbf{x_1}}+\mathbf{A}\overrightarrow{\mathbf{x_2}}
    \end{equation*}
\end{enumerate}
I když se jedná o poměrně triviální fakt, jde o podstatnou myšlenkovou úvahu.
Autor měl například dlouho problém pochopit termín \textit{podobnost matic}. Matice je jenom
pole čísel, proto podobnost nedává žádný smysl, obzvláště pokud podobné matice mají zcela
rozdílná čísla. Jejich podobnost spočívá v tom, že odpovídají stejnému lineárnímu zobrazení
v různých bázích. I další pojmy algebry, například vlastní čísla matice, dávají smysl až
s tímto pozorováním.

\end{document}
